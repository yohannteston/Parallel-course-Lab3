\chapter{Data sharing}

We can assume that $a$ will be equal to 10 after the datasharing program:
\begin{verbatim}
$ ./datasharing 
a= 10 b= 10
a= 10 b= 10
a= 10 b= 10
a= 10 b= 10
\end{verbatim}

As we see every thread modifies its own $a$ locally, so they do not use the first initialization of $a$. Therefore, the $a$ manipulated by the threads is not the global $a$ but a local variable (which was not initialized). 
After adding the $firstprivate$ pragma, we can expect that $a$ will be added $10$ once in each thread, but this time starting from $a = 10$, because the local $a$ is initialized with the value of the global $a$:

\begin{verbatim}
$ ./datasharing 
a= 20 b= 10
a= 20 b= 10
a= 20 b= 10
a= 20 b= 10
\end{verbatim}

We see that $a$ is now initialized to 10 and equals to $20$ at the end because the calculus $a += b$ is made in every thread.
