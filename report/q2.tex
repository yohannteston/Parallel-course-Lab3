\chapter{Data sharing}

We can assume that $a$ will be equal to 10 after the datasharing program:
\begin{verbatim}
$ ./datasharing 
a= 10 b= 10
a= 10 b= 10
a= 10 b= 10
a= 10 b= 10
\end{verbatim}

As we see every thread modifies $a$ locally, so they don't use the first initialization of $a$.
After adding the $firstprivate$ pragma, we can expect that $a$ will get $10$ one time in each thread starting to $a = 10$:

\begin{verbatim}
$ ./datasharing 
a= 20 b= 10
a= 20 b= 10
a= 20 b= 10
a= 20 b= 10
\end{verbatim}

We see that $a$ is now initialized to 10  and equals to $20$ as the calculus $a += b$ is made in every thread.
